\documentclass{article}
\usepackage[utf8]{inputenc}
\usepackage{amsmath}
\usepackage{graphicx}
\usepackage{multicol}
\usepackage[a4paper, total={7in, 9.5in}]{geometry}

\title{WDDS MDO Shenanigans}
\author{Nate DeGoede}
\date{May 2025, prolly not, but let's be optimistic}

\begin{document}

\maketitle
\begin{multicols}{2}

\section{Methodology}
\subsection{Model}
\subsubsection{Hydrodynamics}
\subsubsection{System Dynamics}
\subsubsection{Economics}
Our objective function LCOW is calculated using the AWP from the system dynamics module and the costs calculated from this economic module. We use a simple cost model with costs split into two categories: capital costs (CAPEX) and operational (OPEX). The equation below for LCOW is a modification of the LCOE euqation proposed by the U.S. Department of Energy \cite{LCOE_DOE}
\begin{equation}
    \text{LCOW} = \frac{(\text{FCR}\times\text{CAPEX}) + \text{OPEX}}{\text{AWP}}
\end{equation}
where FCR is the fixed charge rate, 10.8\% based on te assumptions made in the U.S. Department of Energy report \cite{LCOE_DOE}. The CAPEX and OPEX are split into three sections based on the three main sections of design variables: the wave energy converter, the hydraulics, and the desalination plant. It is worth noting that this cost model is not intended to be a perfect representaion of the total system costs, but rather a way to quantify estimated savings between different desings as a way to demonstrate the impact of an MDO approach to this problem. An explanation of the cost model for each section is given below.
\paragraph{Wave Energy Converter}
\paragraph{Hydraulics}
\paragraph{Desalination Plant}
The costs associated with the desalination plant are split into two categories: capital costs and operational costs. The capital costs are the costs associated with the construction of the plant, while the operational costs are the costs associated with the yearly operation of the plant. Cost models for SWRO desalination plants of this scale (thousands of m$^3$/day) are quite rare, and generally either quite poorly documented or implemented. The majority of well documented economic studies are for much larger plants (hundreds of thousands of m$^3$/day) \cite{Slocum2016,Haefner2023,roopexcurve,Wittholz2008}. Studies that focus on the economic modeling of smaller plants generally have poorly documented cost estimates for how to estimate SWRO plant costs \cite{Elkadeem2024,Goekcek2016}. Another study, specific to wave driven desalination using the Desalination Economic Evaluation Program (DEEP 5.1) developed by the International Atomic Energy Agency for a unit CAPEX cost \cite{Yu2022}. However, the DEEP 5.1 model itself is poorly documented \cite{DEEP5manual}, making it hard to know if this number is a good represntation. 
This study instead uses a a cost model based on the cost estimation methodology presented by Voutckov in his book "Desalination Project Cost Estimating and Management" \cite{voutch}. Using the cost curves from sections 4 (CAPEX) and 5 (OPEX), a cost model that can be customized to the specific needs of a given project can be created. Each curve was fitted to simple power fit shown in the equation below
\begin{equation}
    y = a \cdot x^b
\end{equation}
This form was chosen because it ensures there won't be strange behaviors as the low end of the range. Other forms such as polynomial fits could result in negative costs at the low end of the range, which is nonsensical.

\bibliographystyle{ieeetr}
\bibliography{biblio}
\end{multicols}
\end{document}